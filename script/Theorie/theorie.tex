\section{Theoretische Grundlagen}
\subsection{Biometrie}
TORBEN 
Bei der Biometrie handelt es sich um eine Wissenschaft, welche sich mit der Vermessung von menschlichen Merkmalen beschäftigt.
Die Ergebnisse dieser Messungen können dann dazu verwendet werden, Individuen zu beschreiben und zu identifizieren. Dieser Bereich 
der Biometrie wird auch als \textit{biometrische Erkennungsverfahren} beschrieben. Eine andere Facette der Biometrie, die \textit{biometrische Statistik},
beschäftigt sich mit der Auswertung der erfassten Daten um diese zur Analyse zu nutzen.
Mit der biometrischen Statistik, werden wir uns in dieser Studienarbeit beschäftigen, um die Merkmale, die mittels 
Smartphone erfasst werden, auszuwerten und damit Rückschlüsse auf die Emotionen eines Menschen zu ermöglichen.
\subsection{Smartphone}
sollte auch erklärt werden weil wir ja damit primär arbeiten
\subsection{Was sind Emotionen?}
TORBEN
Da wir uns in dieser Arbeit primär mit der Erkennung von Emotionen beschäftigen wollen, ist es wichtig den Begriff Emotion zu definieren.
Schwarzer-Petruck beschreibt in ihrem Werk Emotionen und pädagogische Professionalität eine Emotion als \textit{``ein komplexes
Muster körperlicher und mentaler Veränderungen als Antwort auf eine als persönlich bedeutsam wahrgenommene Situation''}\footcite[S.51 Z.20ff]{Sch13}.
Das Muster umfasst laut Schwarzer-Petruck die Aspekte des kognitiven Prozess, die Gefühle, eine Verhaltensreaktion und eine physiologische Erregung.
\subsection{Grundlagen der Emotionserkennung}
Nach dem nun geklärt ist was unter einer Emotion verstanden wird, stellt sich die Frage wie man diese erkennen kann. Das Problem hierbei ist, dass es eine große Anzahl an Emotionen gibt, laut Hokuma\footcite[Vgl.][Absch. 1]{Hok17} sind es 34.000 unterschiedliche Emotionen. Diese verschiedenen Emotionen lassen sich nur schwer erfassen und unterscheiden, weshalb ein Weg gefunden werden muss die Emotionen einzuteilen. Diese Einteilung wurde bereits von Robert Plutchick vorgenommen und herausgekommen sind dabei acht primäre Emotionen: Freude, Traurigkeit, Akzeptanz, Ekel, Angst, Wut, Überraschung und Erwartung.
Mit diesen acht Emotionen hat Plutchick das Rad der Emotionen gebildet(siehe Abbildung).
\begin{figure}[h]
	\centering
	\includegraphics[width=16cm]{Bilder/wheel-of-emotions.png}
	\caption[Rad der Emotionen - Robert Plutchick]{Rad der Emotionen - Robert Plutchick\footnotemark}
\end{figure}%
\footcitetext[Vgl.][]{Hok17}
\newline
Das Rad stellt die primären Emotionen dabei in Relation, wobei die Kombinationen zwischen zwei Emotionen im Raum zwischen diesen steht und Emotionen die gegensätzlich wirken, z. Bsp. Traurigkeit und Freude, jeweils auch gegenüberliegend auf dem Rad sind. Außerdem wird die Stärke einer Emotion durch deren nähe zum Zentrum des Rads gekennzeichnet, z. Bsp. Wut zu toben \footcite[Vgl.][Absch. Elements of the Wheel]{Hok17}.
In der Literatur gibt es neben dem Model von Plutchick auch das \textit{Gevena Emotion Wheel}. Dieses Modell betrachtet die Emotionen nicht in acht primären Hauptkategorien, sondern unterscheidet zwischen 20 Emotionen anhand von zwei Parametern, die Valenz und die Kontrolle. Die Kontrolle bezeichnet, wie stark Individuum eine Situation kontrollieren kann. Die Valenz sagt aus ob eine Situation für das Individuum eher angenehm oder unangenehm ist. 
Beide Modelle können dafür genutzt werden um Emotionen auszuwerten, wobei hier zu diskutieren ist welches Modell besser geeignet ist. 
\subsection{Welche Möglichkeiten gibt es?}
\subsubsection{Nutzerinteraktionen}
Im Laufe des Alltags verwenden Nutzer ihr Smartphone sehr häufig. Dabei können unteranderem Aspekte wie das Tippverhalte, 
z. Bsp. verwendet der User viele Smileys, Rückschlüsse auf den emotionalen Zustand eines Nutzers ermöglichen.
\subsubsection{Im Smartphone eingebaute Sensoren}
\subsubsection{Zusätzliche Hardware}
Im Rahmen des Projektes wird die Möglichkeit erforscht, mit Hilfe eines Arduinos die Hautleitfähigkeit aufzuzeichnen. 
Diese ist ein großer Faktor bei der Bestimmung von Emotionen und wird unteranderem auch in Lügendetektoren verwendet. 
\begin{figure}[h]
	\centering
	\includegraphics[width=11cm]{Bilder/sensor.jpg}
	\caption{Das ist ein cooler GSR Sensor}
\end{figure}%