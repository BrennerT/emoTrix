\section{Umsetzung des Prototypen emoTrix}
Hier werden Dinge umgesetzt.
\subsection{MockUps}
\subsection{Datenerfassung in Form von Emotionstests}
\subsubsection{GSR-Test}
\subsubsectionauthor{Lukas Seemann}
Die Erfassungsmöglichkeit mit der höchsten Priorität war die Messung der Hautleitfähigkeit mithilfe von EDA- beziehungsweise GSR-Sensoren. Da dies nicht mit im Smartphone enthaltenen Sensoren möglich war, wurde zusätzlich ein Arduino-Mikrocontroller benötigt, um die Messung durchzuführen. Zunächst wird die Entwicklung auf dem Arduino-Board mit allen zusätzlichen Modulen beschrieben. Im Anschluss daran wird thematisiert, wie das Arduino-Board mit der App verbunden wurde. \newline
Für das Projekt wurde ein Arduino UNO R3 Board verwendet. \footcite[Vgl.][]{Ard18} Dieser kann mit Stromzufuhr über ein Netzteil oder per USB betrieben werden. Als Sensor wurde ein GSR Sensor des Grove-Toolkits verwendet\footcite[Vgl.][]{Gro18}, der bereits in Abbildung 5 gezeigt wird. Am Sensor selbst werden die Elektroden für die Finger angebracht. Da ein handelsübliches Arduino UNO R3 Board nicht über den benötigten Anschluss für den Grove GSR-Sensor verfügt, muss zusätzlich noch ein Grove Base Shield angebracht werden. Dieses kann auf das Arduino Board aufgesteckt werden und erweitert es um viele verschiedene Anschlüsse, unter anderem für Sensoren.
\begin{figure}[h]
	\centering
	\includegraphics[width=16cm]{Bilder/arduino.jpg}
	\caption[Arduino UNO R3 (links) und Grove Base Shield]{Arduino UNO R3 (links) und Grove Base Shield\footnotemark}
\end{figure}%
\footcitetext[Bilder von:][]{Sou18, Rei18}
\newline \newline
Mit diesen Komponenten werden die vom Sensor zurückgelieferten Daten an den Arduino geleitet. Von dort aus müssen die Daten, an die mobile Applikation weitergeleitet werden. Aus diesem Grund muss an das Arduino Board ein Bluetooth-Modul angebracht werden, das Daten senden und empfangen kann. Das Empfangen von Daten ist notwendig, um die Messung zu Starten, wohingegen das Senden für die Übermittlung der Sensordaten benötigt wird. Heutige Smartphones verfügen meistens immer über eine Bluetooth-Schnittstelle, aus welchem Grund Bluetooth gut für die Übertragung geeignet ist. Eine weitere Möglichkeit wäre die Übertragung über WiFi gewesen. Das Arduino-Board wurde mit einem HC05-Bluetooth-Modul erweitert, welches Daten senden und empfangen kann. Dieses ist in Abbildung ? zu sehen.
\begin{figure}[h]
	\centering
	\includegraphics[width=7cm]{Bilder/hc05.jpg}
	\caption[HC-05-Bluetooth-Modul für Arduino]{HC-05-Bluetooth-Modul für Arduino\footnotemark}
\end{figure}%
\newline
Die Beschreibung der Entwicklungsarbeiten wird in zwei Teile aufgespalten. Der erste Teil ist der Quellcode des Arduinos, der zweite Teil die Entwicklung des Emotionstest in der emoTrix-App. \newline
In Listing ? ist der Quellcode des Arduinos abgebildet. In Zeile 1 wird die SofwareSerial-Bibliothek eingebunden, die eine Verwendung der Pins des Arduinos für verschiedene Module ermöglicht. In Zeile 2 wird dem Arduino mitgeteilt, dass auf den Pins 10 und 11 ein Bluetooth-Modul angeschlossen ist und eine Konstante (GSR) festgelegt, die auf den Anschluss A0 des Grove Shields verweist, an dem der GSR Sensor angeschlossen ist. \newline
Generell besteht der Programmcode des Arduinos immer aus zwei Bestandteilen: einem Setup-Block und einem Loop-Block. Der Setup-Block wird einmalig beim Einschalten des Arduinos ausgeführt. Danach wird der Loop-Block solange wiederholt, bis der Arduino ausgeschalten wird. \newline \newline 

\begin{lstlisting}[caption={Quellcode des Arduinos},style=Arduino]
#include <SoftwareSerial.h>
SoftwareSerial BTserial (10, 11); const int GSR=A0;
int sensorValue=0; int gsr_average=0;
boolean measuring = false; char BTString;

void setup(){
	BTserial.begin(9600);
}

void loop(){
	BTString = BTserial.read();
	if(BTString == 'S'){
		measuring = true;
	}
	if(BTString == 'F'){
		measuring = false;
	}
	if(measuring){
		long sum=0;
		for(int i=0;i<10;i++){ 
			sensorValue=analogRead(GSR);
			sum += sensorValue; delay(5);
		}
		gsr_average = sum/10;
		BTserial.print(gsr_average); BTserial.println( ";");
	}
}
\end{lstlisting}
In Zeile 7 innerhalb des Setup-Blocks wird die Geschwindigkeit der seriellen Datenübertragung der Ports des Arduinos, die mit dem Bluetooth-Modul verbunden sind. Hierbei wird die Geschwindigkeit auf 9600 Bits pro Sekunde gesetzt.\footcite[Vgl.][]{Ard18b} Dies entspricht der üblich verwendeten Geschwindigkeit und hat in Tests sehr gut funktioniert. \newline
Der Loop-Block beginnt in Zeile 11 mit dem Auslesen der Daten, die über das Bluetooth-Modul empfangen werden. Die Variable \textit{BTString} wird mit diesen Daten beschrieben. Die App muss zum Starten der App den String \textit{S} (für Start) per Bluetooth übertragen. Ist dies der Fall, wird die Variable \textit{measuring} auf true gesetzt. Mit dem String \textit{F} (für Finished) kann die App dem Arduino das Stopsignal für die Messung geben. Demenstprechend wird \textit{measuring} auf false gesetzt. Dies ist in den Zeilen 15 bis 17 umgesetzt. \newline
In Zeile 18 wird über die \textit{measuring}-Variable überprüft, ob gemessen werden soll. Wenn ja, wird eine Variable für die Summe von 10 Messdaten initialisiert. Anschließend werden in Abstand von 5 Millisekunden 10 Messungen durchgeführt. In Zeile 21 wird die eigentliche Messung des Sensors durchgeführt. Der hier verwendete GSR-Sensor liefert als Output einen Integer-Wert, der die Stromspannung auf dem seriellen Port des Arduinos in Volt entspricht. Dieser Wert hängt von der gemessenen Hautleitfähigkeit ab. Die Umrechnung in den Hautleitfähigkeit geschieht dann in der App und wird im Anschluss noch betrieben. Im Arduino-Code selbst werden die Daten überliefert, die auch der Sensor übermittelt.  Alle 10 Messungen werden nach und nach aufaddiert. Dies geschieht in der for-Schleife in den Zeilen 20 bis 23. Anschließend wird die Summe durch 10 geteilt, sodass man den Durchschnitt aller 10 Werte erhält. Dieses Verfahren wird durchgeführt, da die Messdaten des Sensors Schwankungen aufweisen, die dadurch eliminiert werden können. In Zeile 25 wird schließlich der Messwert auf den Port des Bluetooth-Moduls geschickt und damit versendet. Als Trennzeichen zum nächsten Wert wird ein Semikolon angehängt. \newline
Wenn die Messung gestartet wurde, erhält die mobile Applikation also alle 50 Millisekunden vom Arduino per Bluetooth einen Integer-Wert übermittelt. \newline
Die Daten, die der Arduino schickt, müssen innerhalb der App verarbeitet werden. Hierfür wurde eine neue TestPage mit dem Namen GSRPage erstellt. Diese Page dient dazu, sich mit dem Arduino Bluetooth-Modul zu verbinden und die Messung zu starten und zu stoppen. Zusätzlich werden die gemessenen Werte in einem Graphen dargestellt, um zu veranschaulichen, wie sich die Hautleitfähigkeit während der Messung ändert. \newline \newline
In der gsr.ts-Datei sind alle Funktionen hinterlegt, die alle Methoden für die Umsetzug der genannte Funktionalitäten implementiert. In Listing 2 ist ein Ausschnitt aus der Typescript-Datei mit den wichtigsten Funktionen abgebildet. \newline
Die Funktion \textit{startMeasuring} implementiert das Starten der Messung. Für die GSRPage wurde das Ionic-Package Bluetooth Serial hinzugefügt, das Bluetooth-Optionen des Smartphones für die App verfügbar macht. Das Package wird hier mit \textit{this.bluetoothSerial} referenziert. Das Pairen und Verbinden des Smartphones mit dem Bluetooth-Modul ist selbsterklärend (über eine \textit{connect}-Methode) und deswegen im Listing nicht aufgeführt. Nachdem die Verbindung eingerichtet wurde, kann die \textit{startMeasuring}-Funktion aufgerufen werden. Dabei wird in Zeile 2 mit \textit{write} ein String an das verbundene Bluetooth-Modul gesendet. Wie bereits beschrieben muss zum Start der String \textit{S} gesendet werden. Falls ein Fehler auftritt, wird dieser in Zeile 4 geloggt. Wenn alles ordnungsgemäß funktioniert, empfängt der Arduino den String und startet anschließend die Messung.\newline
Die \textit{stopMeasuring}-Funktion funktioniert analog zur \textit{startMeasuring}-Funktion mit dem Unterschied, dass hier der String \textit{F} zum Stoppen der Messung gesendet wird.
\newpage
\begin{lstlisting}[caption={JS Code}, language=JavaScript]
startMeasuring(){
	this.bluetoothSerial.write('S').then((data: any) => { })
		.catch((e) => {
			console.log(e);
		});;
}

stopMeasuring(){
	this.bluetoothSerial.write('F').then((data: any) => {})
		.catch((e) => {
			console.log(e);
		});
}

this.bluetoothSerial.subscribe(";").subscribe(
	function (data){
		self.value = data.substring(0,data.length - 1);
		if(self.time%10 == 0){
			self.addData(self.lineChart,self.time, self.value);
			if(self.time != 0){
				var data: any = {value: self.value, oldValue: self.oldValue};
				self.GsrSensor.onSensorData(data);
			}  
			self.oldValue = self.value;
		}
		self.time++;
	}, function (error){
		console.log(error);
});

\end{lstlisting}
In Zeile 15 ist implementiert, dass das Bluetooth des Smartphones auf einen neu eintreffenden Integer-Wert des Arduinos reagieren kann. Das BluetoothSerial dient dabei als Observable, den die App abonnieren kann. Dies bedeutet, dass die App darauf hingewiesen wird, wenn neue Daten angekommen sind. Mit \textit{subscribe(";")} wird die App jedes Mal informiert, wenn ein Semikolon übertragen wurde. Das Semikolon wurde im Arduino-Code als Trennzeichen eingesetzt und signalisiert, dass ein Integer-Wert abgeschlossen ist. Mit dem nächsten \textit{subscribe} wird bestimmt, was ausgeführt wird, wenn ein Semikolon empfangen wird. Als \textit{data} (Zeile 16) wird immer alles übertragen, was seit dem letzten Ausführen der Funktion übertragen wurde. Es handelt sich also immer um einen Integer-Wert und das Semikolon. In Zeile 17 wird aus diesem Grund das letzte Zeichen der Übertragung abgeschnitten, sodass der \textit{value}-Variable nur der Integer-Wert zugewiesen wird.

\subsection{Auswertung der Testergebnisse}
\subsubsection{Beschreibung der implementierten Kausalitätsregeln}
\subsubsection{Implementierung des Deciders}
\subsection{Benutzeroberfläche der App}
\subsubsection{HomePage}
\subsubsection{GSRPage}
\subsubsection{CameraPage}
\subsubsection{DecisionPage}
