\documentclass[12pt,titlepage,ngerman]{article}

\usepackage[ngerman]{babel}
\usepackage[utf8]{inputenc}
\usepackage{color}
\usepackage[a4paper,lmargin={2,54cm},rmargin={2,54cm},tmargin={2.54cm},bmargin = {2.54cm}]{geometry}
\usepackage{enumitem}
\usepackage{amssymb}
\usepackage{amsthm}
\usepackage{graphicx}
\usepackage{acronym}
\usepackage[backend=bibtex,style=alphabetic]{biblatex}
\addbibresource{Studienarbeit.bib}
\usepackage{setspace}
\usepackage{listings}
\usepackage{mwe}
\usepackage{etoolbox}  
\usepackage{tikz}

\makeatletter
\newcommand{\sectionauthor}[1]{%
	\begin{flushright}
			{\parindent0pt\vspace*{-33pt}%
			\linespread{1.1}\large\scshape#1%
			\par\nobreak\vspace*{-1pt}}
		%Inhalt...
	\end{flushright}
}
\newcommand{\subsectionauthor}[1]{%
	\begin{flushright}
		{\parindent0pt\vspace*{-24pt}%
			\linespread{0.9}\large\scshape#1%
			\par\nobreak\vspace*{-4pt}}
		%Inhalt...
	\end{flushright}
}
\newcommand{\subsectionauthorlong}[1]{%
	\begin{flushright}
		{\parindent0pt\vspace*{-5pt}%
			\linespread{0.9}\large\scshape#1%
			\par\nobreak\vspace*{-4pt}}
		%Inhalt...
	\end{flushright}
}
\newcommand{\subsubsectionauthor}[1]{%
	\begin{flushright}
		{\parindent0pt\vspace*{-21pt}%
			\linespread{0.9}\scshape#1%
			\par\nobreak\vspace*{-5pt}}
		%Inhalt...
	\end{flushright}
}
\newcommand{\subsubsectionauthorlong}[1]{%
	\begin{flushright}
		{\parindent0pt\vspace*{-5pt}%
			\linespread{0.9}\scshape#1%
			\par\nobreak\vspace*{-5pt}}
		%Inhalt...
	\end{flushright}
}
\makeatother

\definecolor{mygreen}{rgb}{0,0.6,0}
\definecolor{mygray}{rgb}{0.47,0.47,0.33}
\definecolor{myorange}{rgb}{0.8,0.4,0}
\definecolor{mywhite}{rgb}{0.98,0.98,0.98}
\definecolor{myblue}{rgb}{0.01,0.61,0.98}
\definecolor{lightgray}{rgb}{.9,.9,.9}
\definecolor{darkgray}{rgb}{.4,.4,.4}
\definecolor{purple}{rgb}{0.65, 0.12, 0.82}

\newcommand*{\FormatDigit}[1]{\ttfamily\textcolor{mygreen}{#1}}
%% https://tex.stackexchange.com/questions/32174/listings-package-how-can-i-format-all-numbers
\lstdefinestyle{FormattedNumber}{%
	literate=*{0}{{\FormatDigit{0}}}{1}%
	{1}{{\FormatDigit{1}}}{1}%
	{2}{{\FormatDigit{2}}}{1}%
	{3}{{\FormatDigit{3}}}{1}%
	{4}{{\FormatDigit{4}}}{1}%
	{5}{{\FormatDigit{5}}}{1}%
	{6}{{\FormatDigit{6}}}{1}%
	{7}{{\FormatDigit{7}}}{1}%
	{8}{{\FormatDigit{8}}}{1}%
	{9}{{\FormatDigit{9}}}{1}%
	{A0}{{\FormatDigit{A0}}}{2}% Following is to ensure that only periods
	{.1}{{\FormatDigit{.1}}}{2}% followed by a digit are changed.
	{.2}{{\FormatDigit{.2}}}{2}%
	{.3}{{\FormatDigit{.3}}}{2}%
	{.4}{{\FormatDigit{.4}}}{2}%
	{.5}{{\FormatDigit{.5}}}{2}%
	{.6}{{\FormatDigit{.6}}}{2}%
	{.7}{{\FormatDigit{.7}}}{2}%
	{.8}{{\FormatDigit{.8}}}{2}%
	{.9}{{\FormatDigit{.9}}}{2}%
	%{,}{{\FormatDigit{,}}{1}% depends if you want the "," in color
	{\ }{{ }}{1}% handle the space
	,%
}


\lstset{%
	backgroundcolor=\color{mywhite},   
	basicstyle=\footnotesize,       
	breakatwhitespace=false,         
	breaklines=true,                 
	captionpos=b,                   
	commentstyle=\color{red},    
	deletekeywords={...},           
	escapeinside={\%*}{*)},          
	extendedchars=true,              
	frame=shadowbox,                    
	keepspaces=true,                 
	keywordstyle=\color{myorange},       
	language=Octave,                
	morekeywords={*,...},            
	numbers=left,                    
	numbersep=5pt,                   
	numberstyle=\tiny\color{mygray}, 
	rulecolor=\color{black},         
	rulesepcolor=\color{myblue},
	showspaces=false,                
	showstringspaces=false,          
	showtabs=false,                  
	stepnumber=2,                    
	stringstyle=\color{myorange},    
	tabsize=2,                       
	title=\lstname,
	emphstyle=\bfseries\color{blue},%  style for emph={} 
}    

%% language specific settings:
\lstdefinestyle{Arduino}{%
	style=FormattedNumber,
	keywords={void},%                 define keywords
	morecomment=[l]{//},%             treat // as comments
	morecomment=[s]{/*}{*/},%         define /* ... */ comments
	emph={HIGH, OUTPUT, LOW, BTserial, GSR, sensorValue, gsr_average, measuring, BTString, sum},%        keywords to emphasize
}
\lstdefinelanguage{JavaScript}{
	keywords={typeof, new, true, false, catch, function, return, null, catch, switch, var, if, in, while, do, else, case, break, let, abstract},
	keywordstyle=\color{blue}\bfseries,
	ndkeywords={class, export, boolean, throw, implements, import, this},
	ndkeywordstyle=\color{darkgray}\bfseries,
	identifierstyle=\color{black},
	sensitive=false,
	comment=[l]{//},
	morecomment=[s]{/*}{*/},
	commentstyle=\color{purple}\ttfamily,
	stringstyle=\color{red}\ttfamily,
	morestring=[b]',
	morestring=[b]"
}

\begin{document}
\begin{titlepage}
	\begin{figure}
		\centering
		\includegraphics[width=11cm]{Bilder/DHBW_MA_Logo.jpg}
		\vspace{1cm}
	\end{figure}%
	\begin{center}
		{\Large\bfseries Angewandte Informatik}\\
		{\Large\bfseries Betriebliches Informationsmanagement}\\
		%
		\vspace{2\baselineskip}
		%
		{\Large\bfseries STUDIENARBEIT}\\
		%
		\vspace{5\baselineskip}
		%
		{\Large Erfassung biometrischer Daten mithilfe von Smartphones zur Emotionsbestimmung}\\
		%
		\vspace{3\baselineskip}
		%
		{\Large Torben Brenner \& Lukas Seemann}\\
		%
		\vspace{7\baselineskip}
		%
		\begin{tabular}{lllllllll}
			Matrikelnummern: & 7731574, 7526630 \\
			Kurs: & TINF15AI-BI \\
			Bearbeitungszeitraum: & 11.09.2017 - 28.05.2018 \\
			Betreuer der Dualen Hochschule: & Prof. Dr. Holger Hofmann \\ 
		\end{tabular}
	\end{center}
\end{titlepage}
\newpage
\pagenumbering{gobble}
\begin{center}
	{\Large \bfseries Ehrenwörtliche Erklärung}
\end{center}
\vspace{1cm}
Ich versichere hiermit, dass die Studienarbeit mit dem Thema: \textit{Erfassung biometrischer Daten mithilfe von Smartphones zur Emotionsbestimmungs} selbstständig verfasst und keine anderen als die angegebenen Quellen und Hilfsmittel benutzt habe.
\newline \newline
Ich versichere zudem, dass die eingereichte elektronische Fassung mit der gedruckten Fassung übereinstimmt. \newline \newline\newline \newline \newline \newline
{\_\_\_\_\_\_\_\_\_\_\_\_\_\_\_\_\_\_\_\_\_\_\_\_\_\_\_\_\_\_\_\_\_\_\_\_\_}\space\space\space\space\space\space\space\space\space\space\space\space\space\space\space\space\space\space\space\space\space\space\space\space\space {\_\_\_\_\_\_\_\_\_\_\_\_\_\_\_\_\_\_\_\_\_\_\_\_\_\_\_\_\_\_\_\_\_\_\_\_\_\_\_\_\_\_\_\_\_\_\_\_\_\_\_} \newline
Ort \space\space\space\space\space\space\space\space\space\space\space\space\space\space Datum \space\space\space\space\space\space\space\space\space\space\space\space\space\space\space\space\space\space\space\space\space\space\space\space\space\space\space\space\space\space\space\space\space\space\space\space\space\space\space\space\space\space\space\space\space\space\space\space\space\space\space\space Unterschrift
\newline \newline \newline \newline
{\_\_\_\_\_\_\_\_\_\_\_\_\_\_\_\_\_\_\_\_\_\_\_\_\_\_\_\_\_\_\_\_\_\_\_\_\_}\space\space\space\space\space\space\space\space\space\space\space\space\space\space\space\space\space\space\space\space\space\space\space\space\space {\_\_\_\_\_\_\_\_\_\_\_\_\_\_\_\_\_\_\_\_\_\_\_\_\_\_\_\_\_\_\_\_\_\_\_\_\_\_\_\_\_\_\_\_\_\_\_\_\_\_\_} \newline
Ort \space\space\space\space\space\space\space\space\space\space\space\space\space\space Datum \space\space\space\space\space\space\space\space\space\space\space\space\space\space\space\space\space\space\space\space\space\space\space\space\space\space\space\space\space\space\space\space\space\space\space\space\space\space\space\space\space\space\space\space\space\space\space\space\space\space\space\space Unterschrift
\newpage
\onehalfspacing
\begin{tikzpicture}[remember picture, overlay]
\node [anchor=north east, inner sep=0pt]  at (current page.north east)
{\includegraphics[height=4cm]{Bilder/DHBW_MA_Logo_Rand.jpg}};	
\end{tikzpicture}
\begin{flushright}
	\vspace{1cm}
	{\bfseries Angewandte Informatik} \\
	\vspace{\baselineskip}
	{\bfseries Erfassung biometrischer Daten mithilfe von Smartphones} \\
	{\bfseries zur Emotionsbestimmung} \\
	\vspace{0.5cm}
	STUDIENARBEIT \\
	Torben Brenner \& Lukas Seemann \\
	TINF15AI-BI
\end{flushright}

\section*{Abstract}
Eine der Hauptfähigkeiten des Menschen ist es, die vom Gegenüber empfundenen Emotionen zu verstehen. Dazu werden neben der Körperhaltung auch die Stimme und andere Anzeichen analysiert. Da mittlerweile immer mehr auch die Interaktion des Menschen mit Maschinen in den Vordergrund rückt, ist es notwendig, dass auch diese die Emotionen ihres menschlichen Gegenübers verstehen lernen. Forschung in diesem Gebiet werden schon seit Längerem durchgeführt, aber die spezielle Nutzung des Smartphones für die Emotionserkennung bietet einiges an Potenzial.\newline \newline
Deshalb war die Untersuchung der Erfassung biometrischer Daten mit dem Smartphone und deren Eignung für die Analyse der menschlichen Emotion das Ziel dieser Studienarbeit. Dazu wurde eine Recherche durchgeführt in der unterschiedliche biometrische Merkmale auf ihre Eignung zur Emotionsanalyse untersucht wurden. Basierend auf den Ergebnissen dieser Recherche wurde eine Architektur entwickelt, die es ermöglicht, verschiedene biometrische Daten in einer mobilen Applikation zu analysieren. Mit dieser Architektur wurde ein Prototyp entwickelt, der den Hautwiderstand und die menschliche Mimik zur Erfassung von Emotionen analysiert und basierend auf diesen Daten auf die vier Emotionen Wut, Trauer, Freude und Überraschung schließt. Der Prototyp lässt sich um beliebige Sensoren erweitern und kann damit auch für Studien und komplexere Emotionsanalysen in Zukunft verwendet werden.
\newpage
\singlespacing
\addcontentsline{toc}{section}{Inhaltsverzeichnis}
\pagenumbering{Roman}
\tableofcontents
\newpage
\section*{Abkürzungsverzeichnis}
\addcontentsline{toc}{section}{Abkürzungsverzeichnis}
\begin{acronym}[*********]
	\acro{App}{Mobile Applikation}
	\acro{AU}{Action Unit}
	\acro{CPU}{Central Processing Unit}
	\acro{CSS}{Cascading Style Sheets}
	\acro{DAG}{Directed acyclic graph}
	\acro{EDA}{Elektrodermale Aktivität}
	\acro{EVM}{Eulersche Videoverstärkung}
	\acro{FACS}{Facial Action Coding System}
	\acro{FEEL}{Facial Expressed Emotion Labeling}
	\acro{GHz}{Gigahertz}
	\acro{GPS}{Global Positioning System}
	\acro{GSR}{Galvanic Skin Response}
	\acro{GUI}{Graphical User Interface}
	\acro{HTML}{Hypertext Markup Language}
	\acro{KI}{Künstliche Intelligenz}
	\acro{LED}{Light-emitting diode}
	\acro{MIT}{Massachusetts Institute of Technology}
	\acro{NFC}{Near Field Communication}
	\acro{PC}{Personal Computer}
	\acro{PIN}{Persönliche Identifikationsnummer}
	\acro{PPG}{Photoplethysmograph}
	\acro{SCSS}{Sassy Cascading Style Sheets}
	\acro{SMS}{Short Message Service}
	\acro{TS}{TypeScript}
	\acro{UML}{Unified Modelling Language}
\end{acronym}
\newpage
\addcontentsline{toc}{section}{Abbildungsverzeichnis}
\listoffigures
\newpage
\addcontentsline{toc}{section}{Tabellenverzeichnis}
\listoftables
\addcontentsline{toc}{section}{Listings}
\lstlistoflistings
\newpage

\pagenumbering{arabic}
\sloppy
\onehalfspacing
\section{Einleitung}
In diesem Kapitel wird zunächst die Motivation für die Studienarbeit beschrieben. Im Anschluss daran werden die Ziele der Arbeit definiert, um  schließlich die Vorgehensweise zur Erreichung dieser Ziele aufzustellen.
\subsection{Motivation}
\subsectionauthor{Lukas Seemann}
Smartphones sind aus dem Alltag vieler Menschen nicht mehr wegzudenken. Nach Prognosen der Statista GmbH nutzen im Jahr 2018 57 Millionen Menschen in Deutschland ein Smartphone. \footcite{Sta18a} Weltweit betrachtet vergrößert sich die Nutzerzahl für 2018 auf ungefähr 2,53 Milliarden Personen. \footcite{Sta18b}
Hierbei muss der Unterschied zwischen Smartphones und normalen Mobiltelefonen, die als Hauptfunktionalität das Telefonieren besitzen, hervorgehoben werden. Die 2,53 Milliarden Smartphone-Nutzer machen circa 53,3\% aller Mobiltelefonnutzer weltweit aus. \footcite{Sta18c} Smartphones unterscheiden sich von Mobiltelefonen in der Anzahl der Funktionalitäten, die bei Smartphones die übliche Nutzung eines Telefons bei Weitem überschreiten. Um diese zusätzlichen Funktionalitäten bereitzustellen, werden in Smartphones heutzutage viele Arten von Sensoren eingebaut und verwendet, um Daten zu erfassen. Hierzu zählen beispielsweise
\begin{itemize}[noitemsep, topsep=0pt]
	\item GPS-Sensoren zur Positionsbestimmung,
	\item Touchscreens zur einfachen Bedienung des Smartphones,
	\item Beschleunigungssensoren zur automatischen Ausrichtung des Bildschirms, 
	\item Fingerabdrucksensoren zur Authentifizierung des Nutzers und
	\item Helligkeitsensoren zur Anpassung der Bildschirmhelligkeit.\footcite[Vgl. ][]{Bie14}
\end{itemize}
Diese Auflistung ist nur ein kleiner Ausschnitt der Technologien, die in der heutigen Zeit verwendet werden. Ein Potenzial, das sich hieraus ergibt jedoch nicht sehr häufig genutzt wird, ist die Erfassung von biometrischen Daten mithilfe dieser Sensoren. Bei biometrischen Daten handelt es sich um menschliche Merkmale, die als Grundlage für verschiedene Arten von Analysen herangezogen werden können. In der Biometrie gängige Verfahren sind beispielsweise
\begin{itemize}[noitemsep, topsep=0pt]
	\item die Pulsmessung,
	\item die Gesichtserkennung und
	\item die Spracherkennung.
\end{itemize} Die Studienarbeit betrachtet verschiedene Möglichkeiten mithilfe von Smartphone-Sensoren und eventuell zusätzlicher Hardware, biometrische Daten zu erfassen, diese zu analysieren und so dieses selten genutzte Potenzial auszuschöpfen.
\subsection{Zielsetzung}
\subsectionauthor{Torben Brenner \& Lukas Seemann}
Das Ziel dieser Studienarbeit ist es, Möglichkeiten zu erkunden, mit Smartphones biometrische Daten zu erfassen. Dabei werden in das Smartphone integrierte Sensoren, über zusätzliche Hardware angeschlossene Sensoren und die Interaktion des Nutzers mit seinem Smartphone betrachtet. Diese erfassten Daten werden anschließend für Analysen verwendet, die Rückschlüsse auf die Emotionen des Nutzers zulassen
Als finales Produkt soll eine mobile Anwendung für Smartphones entstehen, die den Nutzer verschiedene Tests anbietet, anhand denen die aktuelle Gemütslage beziehungsweise die Emotion des Nutzers bestimmt werden können.
\subsection{Vorgehensweise}
\subsectionauthor{Lukas Seemann}
Um die definierten Ziele der Arbeit zu erreichen, unterteilt sich die Arbeit im Folgenden in vier weitere Kapitel. \newline
Im nächsten Kapitel werden zunächst wichtige theoretischen Grundlagen behandelt, die für das Verständnis der Arbeit notwendig sind. Hierzu zählen Definitionen zu den Themen Biometrie, Emotion, Emotionserkennung und die Auswertung von biometrischen Daten. Außerdem werden technische Themen wie die Grundlagen zu Smartphones, Sensoren und mobilen Applikation thematisiert. \newline
Im dritten Kapitel wird das Konzept der mobilen Anwendung beschrieben. Nachdem festgelegt wurde, welche Art von Art von Daten mithilfe des Smartphones erfasst werden sollen, wird geplant, wie die Daten erfasst werden. In diesem Schritt wird auch bestimmt, ob zusätzliche Hardware benötigt wird oder ob ein handelsübliches Smartphone ausreicht. Im Anschluss daran wird konzipiert, nach welchem Prinzip die erfassten Daten ausgewertet werden. Der letzte Schritt ist der Entwurf eines Entscheidungsalgorithmus, der aus den ausgewerteten Daten eine Emotion des Nutzers bestimmen kann. \newline
Im vierten Kapitel steht die Umsetzung des Konzepts als mobile Applikation im Mittelpunkt. Als erstes wird hierbei die Architektur der App beschrieben. Anschließend werden die konkrete Umsetzung im Ionic-Framework thematisiert und somit im Detail auf den geschriebenen Quellcode eingegangen. \newline
Im Schluss wird ein Fazit zum Ergebnis geliefert und weitere mögliche Schritte des Projekts dargestellt.
\newpage
\section{Theoretische Grundlagen}
\subsection{Was sind Emotionen?}
Hier soll der Begriff emotionen erklärt werden. Siehe \cite{Sch13}.
\newpage
\section{Konzept}
Hier wird ein Konzept mit Mock Ups und Architektur entstehen
\subsection{Priorisierung der Erfassungsmöglichkeiten}
\subsectionauthor{Lukas Seemann}
Nachdem im vorherigen Kapitel verschiedene Möglichkeiten vorgestellt wurden, mit denen Anzeichen von Emotionen bei Menschen gemessen werden können, werden nun diese Möglichkeiten priorisiert. In der folgenden Tabelle (Tabelle 1) ist die Priosierung abgebildet. \newline
\begin{table}[h]
	\centering
	\includegraphics[width=14cm]{Bilder/prio.png}
	\caption[Priorisierung der Erfassungsmöglichkeiten]{Priorisierung der Erfassungsmöglichkeiten}
\end{table}%
\newline In der ersten Spalte ist die Priorität dargestellt. Je niedriger die Zahl ist, desto höher ist die Erfassungsmöglichkeit priorisiert. Die Möglichkeiten werden in der Reihenfolge der hier dargestellten Priorisierung thematisiert und letzten Endes in den Prototyp der mobilen Applikation integriert, um Daten zu erfassen. Je nachdem wie viel Zeit die einzelnen Features benötigen, können mehr und mehr Möglichkeiten der Datenerfassung in die App eingebaut werden, wenn sie noch im Zeitrahmen der Studienarbeit umsetzbar sind. Bei den einzelnen Möglichkeiten werden das Indiz, anhand dessen Rückschlüsse auf eine Emotion gemacht werden kann, und ein Sensor, der Daten zum Indiz für die App erfassen soll, aufgelistet. In der letzen Spalte ist festgehalten, ob der benötigte Sensor in den meisten aktuellen Smartphones bereits enthalten ist oder nicht. \newline
Die höchste Priorität hat das Indiz der Hautleitfähigkeit, die mithilfe von GSR- beziehungsweise EDA-Sensoren erfasst werden kann. Diese Art von Sensoren befinden sich nicht in handelsüblichen Smartphones, weshalb man hierzu externe Sensoren mit dem Handy verbinden muss. \newline
...

\subsection{Ausgewählte Erfassungsmöglichkeiten von biometrischen Daten}
\subsection{Übertragung biometrischer Daten in Emotionscores}
\subsection{Auswertung der Scores zur Emotionsbestimmung}
\subsubsection{Kausalitätsregeln}
\subsubsection{Entscheidungsalgorithmus}
\subsectionauthor{Torben Brenner}
Ziel der Anwendung ist es, basierend auf zuvor aufgenommenen Daten eine Entscheidung zu fällen, welche Emotion der Nutzer der 
Anwendung aktuell empfinden könnte. Die Entscheidung muss dabei die verschiedenen Ergebnisse der Auswertungsebene einbeziehen
und aus diesen auf eine Emotion schließen. Deshalb muss eine Einheitliche Datenstruktur entwickelt werden, über die die Auswertungsebene
die Daten zur Verfügung stellt. \\
Die Entscheidung könnte hierbei über ein \textit{Scoring} entstehen. Dieses \textit{Scoring} müsste dabei auf der Auswertungsebene stattfinden, 
wobei jeder der Auswertungsalgorithmen ein \textit{Scoring} für die verschiedenen Emotionen angeben muss. Am Ende könnten z. Bsp. die verschiedenen 
\textit{Scorings} addiert und die Emotion mit dem höchsten \textit{Scoring} ausgewählt werden.
\subsection{Ionic Framework}
\subsubsection{Aufbau und Einsatz des Frameworks}
\subsubsectionauthor{Lukas Seemann}
Ionic ist ein unter der MIT License stehendes Open-Source-Framework, das zur Entwicklung von plattformübergreifenden, mobilen Applikationen dient. Mit Ionic entwickelte Apps sind damit unter anderem auf Endgeräten lauffähig, die die Betriebssysteme Android, iOS und Windows Phone benutzen. \footcite{Ion18a}
\begin{figure}[h]
	\centering
	\includegraphics[width=11cm]{Bilder/ionic.png}
	\caption[Ionic Framework - Logo]{Ionic Framework - Logo\footnotemark}
\end{figure}%
\footcitetext{Wik18}
\newline
Aktuell befindet sich das Framework in der Version 3.9.2\footcite[Vgl. ][]{Ion18b} und befindet sich in stetiger Weiterentwicklung. Das Ionic Framework basiert wiederum auf Angular, einem Framework für die Entwicklung von Web-Applikationen. Dementsprechend nutzen Ionic-Anwendungen in der Web-Entwicklung etablierte Technologien wie HTML 5, CSS und JavaScript. \footcite[Vgl. ][]{Ion18c} Wie auch im Angular Framework, wird auch die Programmiersprache TypeScript verwendet, die auf JavaScript aufbaut, sich in der Syntax sehr stark mit JavaScript ähnelt und zusätzliche Optionen zur Typisierung von Variablen oder Funktionen anbietet. \footcite[Vgl. ][]{Til17} \newline
Ionic-Anwendungen sind im Wesentlichen normale Webanwendungen, die von jedem JavaScript-fähigen Browser ausgeführt werden können. Während mithilfe von Ionic das Frontend der Anwendung festgelegt wird, kann anschließend mit Apache Cordova die Plattformunabhängigkeit umgesetzt werden. Apache Cordova bewirkt, dass sich die Webanwendungen wie native Android-, iOS- oder Windows Phone-Applikationen anfühlen. Egal auf welcher Plattform die Ionic-Anwendung installiert wird, es wird die selbe Code-Basis verwendet. Diese wird dann vor dem Installieren von Cordova so angepasst, dass sie auf den Endgeräten ausgeführt werden können. \footcite{Ion18d}
\subsubsection{Gründe für die Verwendung}

\subsection{Architektur der mobilen Applikation}

\newpage
\section{Umsetzung des Prototypen emoTrix}
Hier werden Dinge umgesetzt.
\subsection{MockUps}
\subsection{Datenerfassung in Form von Emotionstests}
\subsubsection{GSR-Test}
\subsubsectionauthor{Lukas Seemann}
Die Erfassungsmöglichkeit mit der höchsten Priorität war die Messung der Hautleitfähigkeit mithilfe von EDA- beziehungsweise GSR-Sensoren. Da dies nicht mit im Smartphone enthaltenen Sensoren möglich war, wurde zusätzlich ein Arduino-Mikrocontroller benötigt, um die Messung durchzuführen. Zunächst wird die Entwicklung auf dem Arduino-Board mit allen zusätzlichen Modulen beschrieben. Im Anschluss daran wird thematisiert, wie das Arduino-Board mit der App verbunden wurde. \newline
Für das Projekt wurde ein Arduino UNO R3 Board verwendet. \footcite[Vgl.][]{Ard18} Dieser kann mit Stromzufuhr über ein Netzteil oder per USB betrieben werden. Als Sensor wurde ein GSR Sensor des Grove-Toolkits verwendet\footcite[Vgl.][]{Gro18}, der bereits in Abbildung 5 gezeigt wird. Am Sensor selbst werden die Elektroden für die Finger angebracht. Da ein handelsübliches Arduino UNO R3 Board nicht über den benötigten Anschluss für den Grove GSR-Sensor verfügt, muss zusätzlich noch ein Grove Base Shield angebracht werden. Dieses kann auf das Arduino Board aufgesteckt werden und erweitert es um viele verschiedene Anschlüsse, unter anderem für Sensoren.
\begin{figure}[h]
	\centering
	\includegraphics[width=16cm]{Bilder/arduino.jpg}
	\caption[Arduino UNO R3 (links) und Grove Base Shield]{Arduino UNO R3 (links) und Grove Base Shield\footnotemark}
\end{figure}%
\footcitetext[Bilder von:][]{Sou18, Rei18}
\newline \newline
Mit diesen Komponenten werden die vom Sensor zurückgelieferten Daten an den Arduino geleitet. Von dort aus müssen die Daten, an die mobile Applikation weitergeleitet werden. Aus diesem Grund muss an das Arduino Board ein Bluetooth-Modul angebracht werden, das Daten senden und empfangen kann. Das Empfangen von Daten ist notwendig, um die Messung zu Starten, wohingegen das Senden für die Übermittlung der Sensordaten benötigt wird. Heutige Smartphones verfügen meistens immer über eine Bluetooth-Schnittstelle, aus welchem Grund Bluetooth gut für die Übertragung geeignet ist. Eine weitere Möglichkeit wäre die Übertragung über WiFi gewesen. Das Arduino-Board wurde mit einem HC05-Bluetooth-Modul erweitert, welches Daten senden und empfangen kann. Dieses ist in Abbildung ? zu sehen.
\begin{figure}[h]
	\centering
	\includegraphics[width=7cm]{Bilder/hc05.jpg}
	\caption[HC-05-Bluetooth-Modul für Arduino]{HC-05-Bluetooth-Modul für Arduino\footnotemark}
\end{figure}%
\newline
Die Beschreibung der Entwicklungsarbeiten wird in zwei Teile aufgespalten. Der erste Teil ist der Quellcode des Arduinos, der zweite Teil die Entwicklung des Emotionstest in der emoTrix-App. \newline
In Listing ? ist der Quellcode des Arduinos abgebildet. In Zeile 1 wird die SofwareSerial-Bibliothek eingebunden, die eine Verwendung der Pins des Arduinos für verschiedene Module ermöglicht. In Zeile 2 wird dem Arduino mitgeteilt, dass auf den Pins 10 und 11 ein Bluetooth-Modul angeschlossen ist und eine Konstante (GSR) festgelegt, die auf den Anschluss A0 des Grove Shields verweist, an dem der GSR Sensor angeschlossen ist. \newline
Generell besteht der Programmcode des Arduinos immer aus zwei Bestandteilen: einem Setup-Block und einem Loop-Block. Der Setup-Block wird einmalig beim Einschalten des Arduinos ausgeführt. Danach wird der Loop-Block solange wiederholt, bis der Arduino ausgeschalten wird. \newline \newline 

\begin{lstlisting}[caption={Quellcode des Arduinos},style=Arduino]
#include <SoftwareSerial.h>
SoftwareSerial BTserial (10, 11); const int GSR=A0;
int sensorValue=0; int gsr_average=0;
boolean measuring = false; char BTString;

void setup(){
	BTserial.begin(9600);
}

void loop(){
	BTString = BTserial.read();
	if(BTString == 'S'){
		measuring = true;
	}
	if(BTString == 'F'){
		measuring = false;
	}
	if(measuring){
		long sum=0;
		for(int i=0;i<10;i++){ 
			sensorValue=analogRead(GSR);
			sum += sensorValue; delay(5);
		}
		gsr_average = sum/10;
		BTserial.print(gsr_average); BTserial.println( ";");
	}
}
\end{lstlisting}
In Zeile 7 innerhalb des Setup-Blocks wird die Geschwindigkeit der seriellen Datenübertragung der Ports des Arduinos, die mit dem Bluetooth-Modul verbunden sind. Hierbei wird die Geschwindigkeit auf 9600 Bits pro Sekunde gesetzt.\footcite[Vgl.][]{Ard18b} Dies entspricht der üblich verwendeten Geschwindigkeit und hat in Tests sehr gut funktioniert. \newline
Der Loop-Block beginnt in Zeile 11 mit dem Auslesen der Daten, die über das Bluetooth-Modul empfangen werden. Die Variable \textit{BTString} wird mit diesen Daten beschrieben. Die App muss zum Starten der App den String \textit{S} (für Start) per Bluetooth übertragen. Ist dies der Fall, wird die Variable \textit{measuring} auf true gesetzt. Mit dem String \textit{F} (für Finished) kann die App dem Arduino das Stopsignal für die Messung geben. Demenstprechend wird \textit{measuring} auf false gesetzt. Dies ist in den Zeilen 15 bis 17 umgesetzt. \newline
In Zeile 18 wird über die \textit{measuring}-Variable überprüft, ob gemessen werden soll. Wenn ja, wird eine Variable für die Summe von 10 Messdaten initialisiert. Anschließend werden in Abstand von 5 Millisekunden 10 Messungen durchgeführt. In Zeile 21 wird die eigentliche Messung des Sensors durchgeführt. Der hier verwendete GSR-Sensor liefert als Output einen Integer-Wert, der die Stromspannung auf dem seriellen Port des Arduinos in Volt entspricht. Dieser Wert hängt von der gemessenen Hautleitfähigkeit ab. Die Umrechnung in den Hautleitfähigkeit geschieht dann in der App und wird im Anschluss noch betrieben. Im Arduino-Code selbst werden die Daten überliefert, die auch der Sensor übermittelt.  Alle 10 Messungen werden nach und nach aufaddiert. Dies geschieht in der for-Schleife in den Zeilen 20 bis 23. Anschließend wird die Summe durch 10 geteilt, sodass man den Durchschnitt aller 10 Werte erhält. Dieses Verfahren wird durchgeführt, da die Messdaten des Sensors Schwankungen aufweisen, die dadurch eliminiert werden können. In Zeile 25 wird schließlich der Messwert auf den Port des Bluetooth-Moduls geschickt und damit versendet. Als Trennzeichen zum nächsten Wert wird ein Semikolon angehängt. \newline
Wenn die Messung gestartet wurde, erhält die mobile Applikation also alle 50 Millisekunden vom Arduino per Bluetooth einen Integer-Wert übermittelt. \newline
Die Daten, die der Arduino schickt, müssen innerhalb der App verarbeitet werden. Hierfür wurde eine neue TestPage mit dem Namen GSRPage erstellt. Diese Page dient dazu, sich mit dem Arduino Bluetooth-Modul zu verbinden und die Messung zu starten und zu stoppen. Zusätzlich werden die gemessenen Werte in einem Graphen dargestellt, um zu veranschaulichen, wie sich die Hautleitfähigkeit während der Messung ändert. \newline \newline
In der gsr.ts-Datei sind alle Funktionen hinterlegt, die alle Methoden für die Umsetzug der genannte Funktionalitäten implementiert. In Listing 2 ist ein Ausschnitt aus der Typescript-Datei mit den wichtigsten Funktionen abgebildet. \newline
Die Funktion \textit{startMeasuring} implementiert das Starten der Messung. Für die GSRPage wurde das Ionic-Package Bluetooth Serial hinzugefügt, das Bluetooth-Optionen des Smartphones für die App verfügbar macht. Das Package wird hier mit \textit{this.bluetoothSerial} referenziert. Das Pairen und Verbinden des Smartphones mit dem Bluetooth-Modul ist selbsterklärend (über eine \textit{connect}-Methode) und deswegen im Listing nicht aufgeführt. Nachdem die Verbindung eingerichtet wurde, kann die \textit{startMeasuring}-Funktion aufgerufen werden. Dabei wird in Zeile 2 mit \textit{write} ein String an das verbundene Bluetooth-Modul gesendet. Wie bereits beschrieben muss zum Start der String \textit{S} gesendet werden. Falls ein Fehler auftritt, wird dieser in Zeile 4 geloggt. Wenn alles ordnungsgemäß funktioniert, empfängt der Arduino den String und startet anschließend die Messung.\newline
Die \textit{stopMeasuring}-Funktion funktioniert analog zur \textit{startMeasuring}-Funktion mit dem Unterschied, dass hier der String \textit{F} zum Stoppen der Messung gesendet wird.
\newpage
\begin{lstlisting}[caption={JS Code}, language=JavaScript]
startMeasuring(){
	this.bluetoothSerial.write('S').then((data: any) => { })
		.catch((e) => {
			console.log(e);
		});;
}

stopMeasuring(){
	this.bluetoothSerial.write('F').then((data: any) => {})
		.catch((e) => {
			console.log(e);
		});
}

this.bluetoothSerial.subscribe(";").subscribe(
	function (data){
		self.value = data.substring(0,data.length - 1);
		if(self.time%10 == 0){
			self.addData(self.lineChart,self.time, self.value);
			if(self.time != 0){
				var data: any = {value: self.value, oldValue: self.oldValue};
				self.GsrSensor.onSensorData(data);
			}  
			self.oldValue = self.value;
		}
		self.time++;
	}, function (error){
		console.log(error);
});

\end{lstlisting}
In Zeile 15 ist implementiert, dass das Bluetooth des Smartphones auf einen neu eintreffenden Integer-Wert des Arduinos reagieren kann. Das BluetoothSerial dient dabei als Observable, den die App abonnieren kann. Dies bedeutet, dass die App darauf hingewiesen wird, wenn neue Daten angekommen sind. Mit \textit{subscribe(";")} wird die App jedes Mal informiert, wenn ein Semikolon übertragen wurde. Das Semikolon wurde im Arduino-Code als Trennzeichen eingesetzt und signalisiert, dass ein Integer-Wert abgeschlossen ist. Mit dem nächsten \textit{subscribe} wird bestimmt, was ausgeführt wird, wenn ein Semikolon empfangen wird. Als \textit{data} (Zeile 16) wird immer alles übertragen, was seit dem letzten Ausführen der Funktion übertragen wurde. Es handelt sich also immer um einen Integer-Wert und das Semikolon. In Zeile 17 wird aus diesem Grund das letzte Zeichen der Übertragung abgeschnitten, sodass der \textit{value}-Variable nur der Integer-Wert zugewiesen wird.

\subsection{Auswertung der Testergebnisse}
\subsubsection{Beschreibung der implementierten Kausalitätsregeln}
\subsubsection{Implementierung des Deciders}
\subsection{Benutzeroberfläche der App}
\subsubsection{HomePage}
\subsubsection{GSRPage}
\subsubsection{CameraPage}
\subsubsection{DecisionPage}

\newpage
\section{Schluss}
Hier werden wir darauf eingehen was erreicht wurde was nicht und weshalb nicht.
\subsection{Anwendungsszenarien}
% Studien
% Belegung verschiedener Theorien für mögliche Emotionsindizien
\subsection{Fazit}

\subsection{Ausblick}
% Sollte mit KI Learning verbessert werden können 

\newpage
\addcontentsline{toc}{section}{Literaturverzeichnis}
\printbibliography
\newpage

\end{document}