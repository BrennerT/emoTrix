\section{Einleitung}
In diesem Kapitel wird zunächst die Motivation für die Studienarbeit beschrieben. Im Anschluss daran werden die Ziele der Arbeit definiert, um  schließlich die Vorgehensweise zur Erreichung dieser Ziele aufzustellen.
\subsection{Motivation}
\subsectionauthor{Lukas Seemann}
Smartphones sind aus dem Alltag vieler Menschen nicht mehr wegzudenken. Nach Prognosen der Statista GmbH nutzen im Jahr 2018 57 Millionen Menschen in Deutschland ein Smartphone. \footcite{Sta18a} Weltweit betrachtet vergrößert sich die Nutzerzahl für 2018 auf ungefähr 2,53 Milliarden Personen. \footcite{Sta18b}
Hierbei muss der Unterschied zwischen Smartphones und normalen Mobiltelefonen, die als Hauptfunktionalität das Telefonieren besitzen, hervorgehoben werden. Die 2,53 Milliarden Smartphone-Nutzer machen circa 53,3\% aller Mobiltelefonnutzer weltweit aus. \footcite{Sta18c} Smartphones unterscheiden sich von Mobiltelefonen in der Anzahl der Funktionalitäten, die bei Smartphones die übliche Nutzung eines Telefons bei Weitem überschreiten. Um diese zusätzlichen Funktionalitäten bereitzustellen, werden in Smartphones heutzutage viele Arten von Sensoren eingebaut und verwendet, um Daten zu erfassen. Hierzu zählen beispielsweise
\begin{itemize}[noitemsep, topsep=0pt]
	\item GPS-Sensoren zur Positionsbestimmung,
	\item Touchscreens zur einfachen Bedienung des Smartphones,
	\item Beschleunigungssensoren zur automatischen Ausrichtung des Bildschirms, 
	\item Fingerabdrucksensoren zur Authentifizierung des Nutzers und
	\item Helligkeitsensoren zur Anpassung der Bildschirmhelligkeit.\footcite[Vgl. ][]{Bie14}
\end{itemize}
Diese Auflistung ist nur ein kleiner Ausschnitt der Technologien, die in der heutigen Zeit verwendet werden. Ein Potenzial, das sich hieraus ergibt, jedoch nicht sehr häufig genutzt wird, ist die Erfassung von biometrischen Daten mithilfe dieser Sensoren. Bei biometrischen Daten handelt es sich um menschliche Merkmale, die als Grundlage für verschiedene Arten von Analysen herangezogen werden können. In der Biometrie gängige Verfahren sind beispielsweise
\begin{itemize}[noitemsep, topsep=0pt]
	\item die Pulsmessung,
	\item die Gesichtserkennung und
	\item die Spracherkennung.
\end{itemize} Die Studienarbeit betrachtet verschiedene Möglichkeiten mithilfe von Smartphone-Sensoren und eventuell zusätzlicher Hardware, biometrische Daten zu erfassen, diese zu analysieren und so dieses selten genutzte Potenzial auszuschöpfen.
\subsection{Zielsetzung}
\subsectionauthor{Torben Brenner \& Lukas Seemann}
Das Ziel dieser Studienarbeit ist es, Möglichkeiten zu erkunden, mit Smartphones biometrische Daten zu erfassen. Dabei werden in das Smartphone integrierte Sensoren, über zusätzliche Hardware angeschlossene Sensoren und die Interaktion des Nutzers mit seinem Smartphone betrachtet. Diese erfassten Daten werden anschließend für Analysen verwendet, die Rückschlüsse auf die Emotionen des Nutzers zulassen.
Als Teil der Studienarbeit soll eine mobile Applikation als Prototyp entwickelt werden, die den Nutzer verschiedene Tests anbietet, anhand derer die aktuelle Gemütslage beziehungsweise die Emotion des Nutzers bestimmt werden kann.
\subsection{Vorgehensweise}
\subsectionauthor{Lukas Seemann}
Um die definierten Ziele der Arbeit zu erreichen, unterteilt sich die Arbeit im Folgenden in vier weitere Kapitel. \newline
Im nächsten Kapitel werden zunächst wichtige theoretischen Grundlagen behandelt, die für das Verständnis der Arbeit notwendig sind. Zunächst wird allgemein der Begriff der Biometrie und die Definiton von Emotionen thematisiert. Anschließend wird beschrieben, wie mit biometrischen Daten bezüglich der Erfassung und Verarbeitung umgegangen wird. Außerdem werden alle Emotionsindizien, die im Rahmen dieser Studienarbeit betrachet werden, definiert und die Hardware beschrieben, mit denen diese Indizien erfasst werden können. Zuletzt werden in diesem Kapitel verschiedene Erfassungsmöglichkeiten von biometrischen Daten mithilfe des Smartphones und externen Sensoren vorgestellt. \newline
Im dritten Kapitel wird das Konzept der mobilen Anwendung beschrieben, die in der Lage sein soll, biometrische Daten zu erfassen und in Emotionen umzuwandeln. Das Kapitel beginnt mit der Priorisierung der vorgestellten Erfassungsmöglichkeiten, wodurch festgelegt wird, welche Komponenten für eine zweckgetreue Verwendung der App am wichtigsten sind. Anschließend wird beschrieben, wie in der App Sensordaten zu Emotionsannahmen verarbeitet werden sollen. Es werden außerdem die Technologien vorgestellt, mit denen die App entwickelt wird und dann das Backend der App mit Darstellungsmöglichkeitne wie UML-Diagrammen beschrieben. Auch das Frontend der App wird konzipiert und mithilfe von MockUps ein mögliche Benutzeroberfläche der App geplant. \newline
Im vierten Kapitel steht die Umsetzung des Konzepts als mobile Applikation im Mittelpunkt. Zunächst wird die Umsetzung des Backends der App anhand von mehreren Codebeispielen beschrieben. Im Anschluss daran wird thematisiert, wie die einzelnen biometrischen Messungen beziehungsweise Emotionstests implementiert wurden. Des Weiteren beschäftigt sich das Kapitel auch mit der Logik, wie die biometrischen Daten in Emotionen umgewandelt werden. Zuletzt wird die Benutzeroberfläche beschrieben und somit, wie die App bedient werden kann. \newline
Im Schluss werden Anwendungsszenarien beschrieben, in denen die mobile Applikation sinnvoll eingesetzt werden kann. Anschließend wird ein Fazit zum Ergebnis geliefert und weitere mögliche Schritte des Projekts dargestellt.