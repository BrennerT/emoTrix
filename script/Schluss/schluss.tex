\section{Schluss}
Hier werden wir darauf eingehen was erreicht wurde was nicht und weshalb nicht.
\subsection{Anwendungsszenarien}
Die Möglichkeit, mit dem Smartphone Emotionen zu erfassen hat viele Vorteile und mögliche Anwendungsszenarien. Diese verschiedenen Szenarien möchten wir im Folgenden erläutern.
\subsubsection{Studien}
\subsubsectionauthor{Torben Brenner}
Die Verwendung von Emotionserkennung über das Smartphone, ermöglicht es sehr einfach neue Studien in diesem Bereich durchzuführen. Durch das verteilen einer einfachen Installationsdatei könnten Studienteilnehmer ohne großen Aufwand an einer Studie teilnehmen. Denkbar wären hierbei Szenarios, bei denen die Teilnehmer zum Beispiel zehn Minuten täglich einen Test durchführen sollen oder in denen Sie durch eine bestimmte Auswahl von Sensoren dauerhaft analysiert werden. Damit wären Studien über den Emotionalen Verlauf eines Arbeitstags möglich.\newline
Unsere Architektur eignet sich insbesondere für solche Fälle, da es möglich ist eine beliebige Anzahl an Sensoren im Hintergrund arbeiten zu lassen und deren Ergebnisse zusammen zu führen. Im speziellen könnte auch eine Erweiterung stattfinden, in der Nutzer nicht nur analysiert werden sondern in denen diese dann auch die Auswertung überprüfen. So könnte im Auswertungsbildschirm eine Meldung von falschen Analysen mit eingebaut werden, um kontinuierlich Nutzer Feedback zu sammeln. In einer extrem ausgeweiteten Variante davon, könnte sogar ein neuronales Netz eingesetzt werden, um die Gewichtung von Indizien in Form der Kausalitätsregeln zu verbessern. Dabei würde das Training primär von den Nutzern durchgeführt. 
\subsubsection{Wirtschaft}
Auch in der Wirtschaft ist es interessant, was Kunden während dem Einkauf fühlen. Eine gängige Praxis in Call Centern ist zum Beispiel schon das prüfen des emotionalen Zustandes eines Anrufers. So können die Mitarbeiter im Zweifelsfall direkt deeskalierend mit dem Anrufer/-in sprechen.
\subsubsection{Unterstützung von Autisten}
\subsection{Fazit}

\subsection{Ausblick}
% Sollte mit KI Learning verbessert werden können 